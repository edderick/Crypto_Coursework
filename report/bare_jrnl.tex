\documentclass[11pt,a4paper,twoside]{article}
%\documentclass[journal]{IEEEtran}

\usepackage[stable]{footmisc}
\hyphenation{op-tical net-works semi-conduc-tor}

\usepackage[top=2.2cm, bottom=2.2cm, left=2.4cm, right=2.4cm]{geometry} 

\usepackage{listings}
\lstset{basicstyle=\ttfamily, escapechar=\€}

\usepackage{color}
\usepackage{setspace}

\usepackage{amsfonts}
\usepackage{amsmath}

\definecolor{Code}{rgb}{0,0,0}
\definecolor{Decorators}{rgb}{0.5,0.5,0.5}
\definecolor{Numbers}{rgb}{0.5,0,0}
\definecolor{MatchingBrackets}{rgb}{0.25,0.5,0.5}
\definecolor{Keywords}{rgb}{0,0,1}
\definecolor{self}{rgb}{0,0,0}
\definecolor{Strings}{rgb}{0,0.63,0}
\definecolor{Comments}{rgb}{0,0.63,1}
\definecolor{Backquotes}{rgb}{0,0,0}
\definecolor{Classname}{rgb}{0,0,0}
\definecolor{FunctionName}{rgb}{0,0,0}
\definecolor{Operators}{rgb}{0,0,0}
\definecolor{Background}{rgb}{0.98,0.98,0.98}

\lstnewenvironment{python}[1][]{
\lstset{
numbers=left,
numberstyle=\footnotesize,
numbersep=1em,
xleftmargin=1em,
framextopmargin=2em,
framexbottommargin=2em,
showspaces=false,
showtabs=false,
showstringspaces=false,
frame=l,
tabsize=4,
% Basic
basicstyle=\ttfamily\small\setstretch{1},
backgroundcolor=\color{Background},
language=Python,
% Comments
commentstyle=\color{Comments}\slshape,
% Strings
stringstyle=\color{Strings},
morecomment=[s][\color{Strings}]{"""}{"""},
morecomment=[s][\color{Strings}]{'''}{'''},
% keywords
morekeywords={import,from,class,def,for,while,if,is,in,elif,else,not,and,or,print,break,continue,return,True,False,None,access,as,,del,except,exec,finally,global,import,lambda,pass,print,raise,try,assert},
keywordstyle={\color{Keywords}\bfseries},
% additional keywords
morekeywords={[2]@invariant},
keywordstyle={[2]\color{Decorators}\slshape},
emph={self},
emphstyle={\color{self}\slshape},
%
}}{}

\usepackage{paralist}
\usepackage{enumitem}

\begin{document}

\title{Cryptography: Secrecy Through Numbers}
\author{Edward~JF~Seabrook}

\maketitle

\section{Elliptic Curves}
This question has been divided into four parts, labeled A through D.

\subsection{Part a}
The Elliptic curve I was assigned was $y^2 = x^3 + 7x + 1$, over the field $\mathbb{F}_{17}$. 

I began by calculating all the values of $y^2$:

\begin{table}[h]
\centering
\begin{tabular}{llllllllllllllllll}
$y$   & 0 & 1 & 2 & 3 & 4  & 5 & 6 & 7  & 8  & 9  & 10 & 11 & 12 & 13 & 14 & 15 & 16 \\
$y^2$ & 0 & 1 & 4 & 9 & 16 & 8 & 2 & 15 & 13 & 13 & 15 & 2  & 8  & 16 & 9  & 4  & 1 
\end{tabular}
\end{table}

Then, using two lines of python:

\begin{python}
for x in range(0, 16):
    print (x**3 + 7*x + 1) % 17 
\end{python}

I produced a table of $y^2$ values, which I mapped back to $y$ values.\

\begin{table}[h]
\centering
\begin{tabular}{lll}
$x$ & $y^2$ & $y$     \\
0   & 1     & 1 or 16 \\
1   & 9     & 3 or 14 \\
2   & 6     & --      \\
3   & 15    & 10 or 7 \\
4   & 8     & 5 or 12 \\
5   & 8     & 5 or 12 \\
6   & 4     & 15 or 2 \\
7   & 2     & 6 or 11 \\
8   & 8     & 5 or 12 \\
9   & 11    & --      \\
10  & 0     & 0       \\
11  & 15    & 10 or 7 \\
12  & 11    & --      \\
13  & 11    & --      \\
14  & 4     & 15 or 2 \\
15  & 13    & 8 or 9  \\
16  & 10    &        
\end{tabular}
\end{table}

The points on my curve are: (0,1), (0,16), (1, 3), (1, 14), (3, 10), (3, 7),
(4, 5), (4, 12), (5, 5), (5,12), (6, 15), (6, 2), (7, 6), (8, 5), (8, 12), (10,
0), (11, 10), (11,7), (14, 5), (14, 2), (15, 8), (15, 9) and the point at
infinity. 

The order of my group is twenty four. 

\subsection{Part b}
Yes, my group is cyclic.

The group has order of twenty four, meaning it can have one of the following
structues: $\mathbb{C}_{24}$, $\mathbb{C}_{12}\times\mathbb{C}_{2}$,
$\mathbb{C}_{6}\times\mathbb{C}_{4}$, $\mathbb{C}_{6}\times\mathbb{C}_{4}$,
$\mathbb{C}_{6}\times\mathbb{C}_{2}\times\mathbb{C}_{2}$,
$\mathbb{C}_{3}\times\mathbb{C}_{4}\times\mathbb{C}_{2}$, or
$\mathbb{C}_{3}\times\mathbb{C}_{8}$


\subsection{Part c}
Elliptic curves have the interesting property of that any two points on the
line will intersect the line again at another point. This can be used as a
group operator, and used to generate a group. For some reason this new group
can be nice and secure.  

\subsection{Part d}
To encyrpt a message using elliptic curve cryptography, I have chosen to use a
larger elliptic curve, as the curve that I was assigned has a low order, so
will not be secure. 

I wrote Python code to encyrpt the message using el gammal encryption. 

\vfill
\pagebreak
\section{Vigenere Cipher}
We were given the following message, and instructed to decrypt it:

\begin{quote}
\small
\raggedright
qhc jeqpaeb srxrrp hcoe pbccktjv hyoduxrc qrmgalp hyse yqtpxcrbd ree yqtcktgln
mc gmsepkmcktq xnb oeqbapzhcos mke mc tfb myfn alnabrlp iq qhyq ilqeeoarbd
afrarirp il jijftyoy mo cpftgzaj fndoaqqrsztsoe ympjfcyqimks aluja bc
jajfcgluqiy kxngmujxtca dsoild tfb mykudxcrrrgkg nooabsq thgzh mctck tyheq
mlyze yyrmxd flwcsep pilze reepb hyse zbel ko pbpmotca hyoduxrc qrmgalp il
mryztgze wbt jftrie gp kllwl xbmrt flw qrcf x tpljyk wmrlb iomh lghe ykd flw
fxrb qo gjpjbmckt mke gk ppxcrfcc lnc bxyjpjb iq aonxnr qrmgal qhgp cyk bc rsca
tm zokmrmjiqb tfb sczupftw lf y jeykildfsi rcxl ulrja tyogcq wfflc xvmfdgkg
bbtcztgln zv fskcrfolxl rbsrfne xs ublj xs roohxn bbtcztgln kbcfxngpmq puae
tpljyks axn zb uqbd rl eqqaziiqe a ffdbbn qfdc zhyknci il xn mqhcowgpe qfdc
zhyknci rcpiqqalq dcpiek tffs roohxn bleq kor zhykgc qhc ioefc txlsb od xnw
darb bsq ilptcxd aealdeq lnjv tfb pmtep mrmcijb od qwm darbs yk etxlsxtmo wfl
iq kor xwyoe mc tfb tpljyk cyknmq arqaah tfb tpljyk dcpiek uqfne zokjol pibb
cfxnlbl yqtyzkq qhc lwlbr mc tfb tpljyk hmtetbr axn spe ffs ikouiebde mc tfb
tpljyk pmtep jobbl rl eqqaziiqe a ffdbbn qfdc zhyknci tfxt pblgxbjv lcxkq lur
peaoer hewp tffs gp tfb ela od qhc jeqpaeb 
\end{quote}

My initial guess was that this cipher was a monoalphabetic substitution cipher,
since the ciphertext appears to have spaces in roughly the correct places for
it to be natural language. 

I began solving this cipher by calculating the Index of Coincidence (IC) for
the text. The IC is a weighted sum of the frequency of the letters, and can be
used to determine if a ciphertext has the same frequency distribution as a given
language. If it has the same distribution, then the cipher is a simple
monoalphabetic substitution cipher. The IC is calculated as:

$$
\textup{IC} =  \frac{\sum_{i=1}^{c} n_i(n_i -1)}{N(N - 1)/c}
$$

Where $N$ is the length of the text. $n_1$ to $n_c$ are the counts
for each of the possible characters. And $c$ is the number of
letters in the alphabet. 

To do this I used code I wrote myself that can be found in Appendix
\ref{IC_code}. I obtained a value of 1.128, which is quite different from the
IC for English of 1.73. From this I concluded that the cipher used is not a
simple monoalphabetic substitution. 

Next I attempted to iterated through different strides (number of letters to
skip when sampling), to see if the cipher used was dependent on the position in
the text. A stride of 2 did not produce a better result, yielding 1.126 and
1.129 starting from characters 0 and 1 respectively.  

Fortunately, using a stride of 3 returned a much better set of values: 1.844,
1.759 and 1.80; for starting character positions of 0, 1 and 2 respectively.
Although this is not exactly equal the expected values for English, I felt that
for the size of the messages (total length of the message divided by three),
they indicated that it was worth investigating.

Given that the substitution was a poly-alphabetic substitution with stride of
three, I began by treating the cipher as a Viginere cipher. On inspection, the
triad 'qhc' jumped out at me as a repeating pattern. Due to the commonness of
the word, I figured that the word could be `the'. If this was correct, then the
Viginere key would be 'xay', resulting in a shifts of 23, 0 and 2.

Using the following python code found in Appendix \ref{vig_code}, I generated
the answer to the question. The answer that I obtained was: 

\begin{quote}
\small
\raggedright
the message starts here recently hardware trojans have attracted the attention
of governments and researchers one of the main concerns is that integrated
circuits in military or critical infrastructure applications could be
maliciously manipulated during the manufacturing process which often takes
place abroad however since there have been no reported hardware trojans in
practice yet little is known about how such a trojan would look like and how
hard to implement one in practice one example is dopant trojan this can be used
to compromise the security of a meaningful real world target while avoiding
detection by functional testing as well as trojan detection mechanisms such
trojans can be used to establish a hidden side channel in an otherwise side
channel resistant design this trojan does not change the logic value of any
gate but instead changes only the power profile of two gates an evaluator who
is not aware of the trojan cannot attack the trojan design using common side
channel attacks the owner of the trojan however can use his knowledge of the
trojan power model to establish a hidden side channel that reliably leaks out
secret keys this is the end of the message
\end{quote}
 


\vfill
\pagebreak
\section{Two Part Encryption}
The file ``secret.hex'' contained a secret message. The message had been
encyrpted in two stages, the first stage was unknown, and the second stage was
a simple XOR based cipher. We were given the first two characters of the
intermediate text --`Hg'.

To decrypt the message, the two stages had to be performed in reverse order:
first the XOR stage, followed by the unknown cipher stage. 

I began by creating a python script that first reads in the secret message, and
then retrieves the key for the known portion of the message. Given that we only
had two letters of the intermediate text, I guessed that the key length of the
message is two, and repreated the key that I had retrieved for the length of
the message to retrieve some text. The Python code that I used to decode the
message into it's intermediate form is as follows \ref{XOR_code}.


This resulted in the followin intermediate text: 

\begin{quote}
\small
\raggedright
Hgt ltsbuba pm dumt uy hgt lpyh wqpmpxbr sbr tdxyuct lvyhtqv pm hgtl sdd,
xbobpib hp tctb hgt aqtshtyh lubry.  Yxqtdv sbvpbt igp htddy vpx hgtv gsct hgt
sbyitq uy npouba, lsr pq yulwdv luyhsotb.  Hgtqt sqt lsbv hgubay hgsh lsot dumt
ipqhg gpdruba pb hp sbr yscpxquba. Exh dumt uy xbwqtrukhsedt sbr it sqt pmhtb
lvyhtquty tctb hp pxqytdcty.  It hgubo yxkktyy, gswwubtyy, gtdwuba phgtqy, pq
yxqwsyyuba pxqytdcty iudd lsot dumt ipqhg ducuba, exh it ksb sdisvy et iqpba pq
mqxyhqshtr ev tctbhy. Hguy uy s qsbrpl wgqsyt.  Wgudpypwgtqy gsct s dph hp ysv
sepxh hgt csdxt pm sdd hgtyt hgubay, sbr s duhhdt dtyy hp ysv sepxh pbt pm hgt
lpyh csdxsedt hgubay pm sdd: dpct. Yp it ksb et kdtsq tbpxag sepxh igsh uh
ltsby mpq dumt hp gsct ltsbuba sbr csdxt, exh igtb it wxh rpib pxq wgudpypwgv
eppoy sbr skhxsddv ath pb iuhg ducuba, ltsbuba sbr csdxt ksb et tdxyuct. Ducuba
itdd uy lpqt sqh hgsb ykutbkt pq wgudpypwgv.  Hgtqtmpqt,  hgt pbdv ytbyt it ksb
lsot pm hgt urts hgsh dumt gsy ltsbuba uy hgsh hgtqt sqt yplt qtsypby hp duct
qshgtq hgsb hp rut, sbr hgpyt qtsypby sqt hp et mpxbr ub hgt ducuba pm dumt
uhytdm.
\end{quote}

By simple inspection, this looked like it was likely to be the output of a
basic cipher -- the message contains only alphabetic characters, and there are
spaces evenly distributed throughout the text. Although this could have just
been by chance, it definitely looked like a good lead to follow.

As I believed this text looked like natrual language, I decided the next step was
to calculate the Index of Coincidence (IC) for this text. 

The value calculated for the IC was ``1.7086'', which is very close to that of
the English language (1.73). This finding confirmed my belief that the cipher
used was some form of monoalphabetic substitution.  

By copying the intermediate text into {\tt vim}, and using the search and
replace functionality (s/x/Y/g), I began substituting letters with likely
substitutions. I started basing my substitions on the frequencies of the
letters; i.e. beginning by replacing `t' ,the most common letter in the text,
with `E', the most common letter in English. Once I had replaced a few letters,
it became obvious what words in the text could potentially be, so I based my
substitutions on this. 

Eventually, I managed to obtain the following message. As the message, makes
sense to me in English, I am confident that this passage is the correct
plaintext. 

\begin{quote}
\small
\raggedright
The meaning of life is the most profound and elusive mystery of them all,
unknown to even the greatest minds. Surely anyone who tells you they have the
answer is joking, mad or simply mistaken.  There are many things that make life
worth holding on to and savouring. But life is unpredictable and we are often
mysteries even to ourselves. We think success, happiness, helping others, or
surpassing ourselves will make life worth living, but we can always be wrong or
frustrated by events.  This is a random phrase.  Philosophers have a lot to say
about the value of all these things, and a little less to say about one of the
most valuable things of all: love. So we can be clear enough about what it
means for life to have meaning and value, but when we put down our philosophy
books and actually get on with living, meaning and value can be elusive. Living
well is more art than science or philosophy. Therefore,  the only sense we can
make of the idea that life has meaning is that there are some reasons to live
rather than to die, and those reasons are to be found in the living of life
itself.
\end{quote}

\vfill
\pagebreak
\section{Headline Puzzles}
The headline puzzles were a cryptographic puzzle based around five newspaper
headlines from around the time where the puzzle was published. The 

We were asked to solve the following set of headlines: 

\begin{enumerate}[noitemsep]
\item YNTS QHABT YBK KJVT NR ORLSJN HCTCYA HQYKJV CYOCMBYNT 
\item GXRYK SXRKVWNRNIO YJVONHB NH VH KXASH OAXBBJNHB WNHB 
\item KSXXMT, FVTS SVJYMBF CFI EI BNSYYC JTMKEID 
\item AXITUL PGGTXLW VGA OCXFT AUMCAL VAGH RXDKQPUR PXDM
\item HQUSESTYY TBDSPKTTY YTT ERYHURBRWCVRPW RW JCBRSKJURTWESK DPSRHRTY
\end{enumerate}

I started by trying to decode at least one of the messages. I started by
picking a long word, and trying to find a match for this word based on the
pattern of the letters. 

I wrote this python script to check the pattern of a letters words against an
UK English language dictionary that I downloaded from
http://www.karamasoft.com/ultimatespell/dictionary.aspx. This dictionary file
contained 126,850 words. 

Using this script to find possible matches for ``ERYHURBRWCVRP'' returned only
one result, ``discrimination''. From here, many of the letters in this puzzle
could already be determined. I substituted this word back into my script, and
the words 'policies', 'employees', 'see' and in popped out. 

Substituting these new words back in, resulted in the following phrase: 

"cQrldless employees see discrimination in JamilyJriendly policies"

Although my dictionary didn't contain the phrase `familyfriendly', I was able
to work this out using my intuition.`cQrldless' didn't seem to have any
matching words. It appears that there has been an error here, and given the
context the correct plain text word should be `childless'. 

Using this same method I was able to deduce the other messages as:
\begin{enumerate}[noitemsep]
\item Puzzle 1: ``ntsb urges new ways to combat rising runway incidents''
\item Puzzle 2: ``dutch authorities closing in on human smuggling ring''
\item Puzzle 4: ``childless employees see discrimination in familyfriendly policies''
\end{enumerate}

Puzzles 4 \& 5 lacked any long words that had few enough possible answers from
which to produce an educated guess. 

\begin{lstlisting}
   a b c d e f g h i j k l m n o p q r s t u v w x y z
1. J S O M B   A   C       L Y R     H T N Q   K   V   
2. S   Y G I   B K N     J A H V     W O R X           
3.                                                     
4.                                                     
5. C   H E T J   Q R     S B W P D   U Y V         K   
   a b c d e f g h i j k l m n o p q r s t u v w x y z
\end{lstlisting}
                   
Headline One vs Five

\begin{lstlisting}
                  Z
                  X
                  F J O P D E T V 
                  G I R U M B N W  
                  A C H Q L S Y K 
                F J O P D E T V 
                G I R U M B N W  
                A C H Q L S Y K 
              F J O P D E T V 
              G I R U M B N W
              A C H Q L S Y K
            F J O P D E T V 
            G I R U M B N W 
            A C H Q L S Y K 
          F J O P D E T V  
      ... G I R U M B N W X A C H Q L S Y K F J O P D E T V Z G ...
          A C H Q L S Y K F J
        F J O P D E T V Z G    
        G I R U M B N W X 
        A C H Q L S Y K
      F J O P D E T V 
      G I R U M B N W  
    X A C H Q L S Y K
    F J O P D E T V 
    G I R U M B N W  
    A C H Q L S Y K
                  Z
\end{lstlisting}

Headline One vs Two
        
\begin{lstlisting}
      U Q D M L E B S T N Y V W K Z X F G A J I C O R H P U Q D M
      X F G A J I C O R H P U Q D M L E B S T N Y V W K   X F G A 
      L E B S T N Y V W K   X F G A J I C O R H P U Q D M L E B S       
\end{lstlisting}

Headline Two vs Five 

\begin{lstlisting}
           R U M B N W X A C H Q L S Y K F J O P 
 R U M B N W X A C H Q L S Y K F J O P D E T V 
\end{lstlisting}


\begin{lstlisting}
   a b c d e f g h i j k l m n o p q r s t u v w x y z
1. J S O M B G A P C I Z E L Y R U D H T N Q W K F V X 
2. S C Y G I E B K N T D J A H V Z F W O R X U Q L P M 
3.                                                     
4.                                                     
5. C N H E T J I Q R O F S B W P D L U Y V M Z X A K G 
   a b c d e f g h i j k l m n o p q r s t u v w x y z
\end{lstlisting}


\begin{lstlisting}
Keys 
KEY: G I R U M B N W X A C H Q L S Y K F J O P D E T V Z

  LEBSTNYVWKZXFGAJICORHPUQDM

1 EBSTNYVWKZXFGAJICORHPUQDML
2 JICORHPUQDMLEBSTNYVWKZXFGA
3 XFGAJICORHPUQDMLEBSTNYVWKZ
4 PUQDMLEBSTNYVWKZXFGAJICORH
5 STNYVWKZXFGAJICORHPUQDMLEB
\end{lstlisting}

My final answers for this question were: 


\begin{enumerate}[noitemsep]
\item NTSB URGES NEW WAYS TO COMBAT RISING RUNWAY INCIDENTS
\item DUTCH AUTHORITIES CLOSING IN ON HUMAN SMUGGLING RING
\item DOLLAR EURO OUTPACE YEN IN CHOPPY TRADING
\item RIPKEN LOOKING FOR QUICK RETURN FROM DISABLED LIST
\item CHILDLESS EMPLOYEES SEE DISCRIMINATION IN FAMILYFRIENDLY POLICIES
\end{enumerate}

Once the correct answers have been obtained, it is possible to search the Web,
and retrieve the correct answers from an NSA leaflet. The NSA solution goes on
to uncover the setting, the Key and the Hat: 

Setting: TOADY Key: SYCOPHANT Hat: BOOTLICKER


% references section

% can use a bibliography generated by BibTeX as a .bbl file
% BibTeX documentation can be easily obtained at:
% http://www.ctan.org/tex-archive/biblio/bibtex/contrib/doc/
% The IEEEtran BibTeX style support page is at:
% http://www.michaelshell.org/tex/ieeetran/bibtex/
%\bibliographystyle{IEEEtran}
% argument is your BibTeX string definitions and bibliography database(s)
%\bibliography{IEEEabrv,../bib/paper}
%
% <OR> manually copy in the resultant .bbl file
% set second argument of \begin to the number of references
% (used to reserve space for the reference number labels box)
%\begin{thebibliography}{1}
%\end{thebibliography}

\vfill
\pagebreak
\section*{Appendix} 
\appendix 
\section{Vigenere Code}
\label{vig_code}
\begin{python}
import string

def shift(c, key):
    n = ord(c) - ord('a')
    k = ord(key) - ord('a')
    return chr((n - k) % 26 + ord('a'))

key = "xay"
answer = ""

with open('ciphertext.txt'):
    n = 0
    for c in s:
        if c in string.ascii_lowercase:
            answer += shift(c, key[n%len(key)])
            n += 1
        else:
            answer += c
        
print answer
\end{python}

\section{IC Code}
\label{IC_code}
\begin{python}
from __future__ import division
import string

with open("intermediate.txt") as f: 
    s = f.read()

    freq = {}

    for c in string.ascii_uppercase:
        freq[c.upper()] = 0

    for c in s:
        if c.upper() in string.ascii_uppercase:
            freq[c.upper()] += 1
                           
    N = sum([n for n in freq.values()])
    c = len(string.ascii_uppercase)

    numerator = sum([(n * (n-1)) for n in freq.values()]) 
    denominator = ((N * (N-1))/c)

    IC = numerator / denominator 

    print IC
\end{python}

\section{XOR Code}
\label{XOR_code}
\begin{python}
with open("secret.hex") as f:
    ciphertext = f.read()

    k1 = chr(ord(ciphertext[0:1]) ^ ord("H"))
    k2 = chr(ord(ciphertext[1:2]) ^ ord("g"))
    keystream = (k1 + k2) * (len(ciphertext) / 2)

    plaintext = "".join(
            [chr(ord(k) ^ ord(c)) 
             for k, c in zip(keystream, ciphertext)]
    )

    print plaintext
\end{python}

\section{Headline Code}
\label{headline_code}
\begin{python}
import string


raw_sentence =   """YNTS QHABT YBK KJVT NR ORLSJN 
                 HCTCYA HQYKJV CYOCMBYNT"""
known_sentence = """YNTS QHABT YBK KJVT NR ORLSJN 
                 HCTCYA HQYKJV incident"""


def pattern(s):
    seen = {}
    out = ""
    x = 'a'
    for c in s:
        if c in seen.keys():
            out += seen[c]
        else:
            out += x
            seen[c] = x
            x = chr(ord(x)+1)
    return out

def words_match(a, b):
    for c, d in zip(a, b):
        if c in string.ascii_lowercase:
            if c != d:
                return False
    return True

def find_word(word, pattern_dic):
    word_list = []
    for i, (w, p) in enumerate(pattern_dict):
        if p == pattern(word) and words_match(word, w):
            word_list.append(w)
    return word_list

def expand_known_characters(raw, known):
    out = raw
    for r,k in zip(raw, known): 
      if k in string.ascii_lowercase:
          out = string.replace(out, r, k)
    return out

with open("en-GB.dic") as word_list:
    dictionary = [w.lower() for w 
                  in word_list.read().split("\r\n") 
                  if "'" not in w]
    pattern_dict = [(w, pattern(w)) for w in dictionary]

sentence = expand_known_characters(raw_sentence, 
                                   known_sentence)
print "\"{}\"\n".format(sentence)

guesses = {}

for w in sentence.split(" "):
    guesses[w] = find_word(w, pattern_dict)

guesses = sorted(guesses.iteritems(), key=lambda x:len(x[1]))

for word in guesses:
    print "{} -- {}\n".format(word[0], word[1])
    total += len(word[1])
\end{python}

%\vfill

% that's all folks
\end{document}
