\documentclass[12pt,a4paper,twoside]{article}
%\documentclass[journal]{IEEEtran}

\usepackage{paralist}
\usepackage[stable]{footmisc}
\hyphenation{op-tical net-works semi-conduc-tor}

\usepackage[top=2.4cm, bottom=2.4cm, left=2.95cm, right=2.95cm]{geometry} 

\usepackage{listings}
\lstset{basicstyle=\ttfamily, escapechar=\%}

\usepackage{color}
\usepackage{setspace}

\usepackage{amsmath}

\definecolor{Code}{rgb}{0,0,0}
\definecolor{Decorators}{rgb}{0.5,0.5,0.5}
\definecolor{Numbers}{rgb}{0.5,0,0}
\definecolor{MatchingBrackets}{rgb}{0.25,0.5,0.5}
\definecolor{Keywords}{rgb}{0,0,1}
\definecolor{self}{rgb}{0,0,0}
\definecolor{Strings}{rgb}{0,0.63,0}
\definecolor{Comments}{rgb}{0,0.63,1}
\definecolor{Backquotes}{rgb}{0,0,0}
\definecolor{Classname}{rgb}{0,0,0}
\definecolor{FunctionName}{rgb}{0,0,0}
\definecolor{Operators}{rgb}{0,0,0}
\definecolor{Background}{rgb}{0.98,0.98,0.98}

\lstnewenvironment{python}[1][]{
\lstset{
numbers=left,
numberstyle=\footnotesize,
numbersep=1em,
xleftmargin=1em,
framextopmargin=2em,
framexbottommargin=2em,
showspaces=false,
showtabs=false,
showstringspaces=false,
frame=l,
tabsize=4,
% Basic
basicstyle=\ttfamily\small\setstretch{1},
backgroundcolor=\color{Background},
language=Python,
% Comments
commentstyle=\color{Comments}\slshape,
% Strings
stringstyle=\color{Strings},
morecomment=[s][\color{Strings}]{"""}{"""},
morecomment=[s][\color{Strings}]{'''}{'''},
% keywords
morekeywords={import,from,class,def,for,while,if,is,in,elif,else,not,and,or,print,break,continue,return,True,False,None,access,as,,del,except,exec,finally,global,import,lambda,pass,print,raise,try,assert},
keywordstyle={\color{Keywords}\bfseries},
% additional keywords
morekeywords={[2]@invariant},
keywordstyle={[2]\color{Decorators}\slshape},
emph={self},
emphstyle={\color{self}\slshape},
%
}}{}



\begin{document}

\title{Cryptography: Secrecy Through Numbers}
\author{Edward~JF~Seabrook}

\maketitle

\section{Ellitic Curves}
This question has been divided into four parts, labeled A through D.

\subsection{Part a}
The order of my group is twenty four. 

\subsection{Part b}
Yes, of course it is. 

\subsection{Part c}
Elliptic curves have the interesting property of that any two points on the
lien will intersect the line again at another point. This can be used as a
group operator, and used to generate a group. For some reason this new group
can be nice and secure.  

\subsection{Part d}
Answer here. 


\section{Vigenere Cipher}
We were given the following message, and told to decrypt it:


\begin{lstlisting}
qhc jeqpaeb srxrrp hcoe pbccktjv hyoduxrc qrmgalp hyse
yqtpxcrbd ree yqtcktgln mc gmsepkmcktq xnb oeqbapzhcos mke mc
tfb myfn alnabrlp iq qhyq ilqeeoarbd afrarirp il jijftyoy mo
cpftgzaj fndoaqqrsztsoe ympjfcyqimks aluja bc jajfcgluqiy
kxngmujxtca dsoild tfb mykudxcrrrgkg nooabsq thgzh mctck tyheq
mlyze yyrmxd flwcsep pilze reepb hyse zbel ko pbpmotca
hyoduxrc qrmgalp il mryztgze wbt jftrie gp kllwl xbmrt flw
qrcf x tpljyk wmrlb iomh lghe ykd flw fxrb qo gjpjbmckt mke gk
ppxcrfcc lnc bxyjpjb iq aonxnr qrmgal qhgp cyk bc rsca tm
zokmrmjiqb tfb sczupftw lf y jeykildfsi rcxl ulrja tyogcq
wfflc xvmfdgkg bbtcztgln zv fskcrfolxl rbsrfne xs ublj xs
roohxn bbtcztgln kbcfxngpmq puae tpljyks axn zb uqbd rl
eqqaziiqe a ffdbbn qfdc zhyknci il xn mqhcowgpe qfdc zhyknci
rcpiqqalq dcpiek tffs roohxn bleq kor zhykgc qhc ioefc txlsb
od xnw darb bsq ilptcxd aealdeq lnjv tfb pmtep mrmcijb od qwm
darbs yk etxlsxtmo wfl iq kor xwyoe mc tfb tpljyk cyknmq
arqaah tfb tpljyk dcpiek uqfne zokjol pibb cfxnlbl yqtyzkq qhc
lwlbr mc tfb tpljyk hmtetbr axn spe ffs ikouiebde mc tfb
tpljyk pmtep jobbl rl eqqaziiqe a ffdbbn qfdc zhyknci tfxt
pblgxbjv lcxkq lur peaoer hewp tffs gp tfb ela od qhc jeqpaeb 
\end{lstlisting}


\section{Two Part Encryption}
The file ``secret.hex'' contained a secret message. The message had been
encyrpted in two stages, the first stage was unknown, and the second stage was
a simple XOR based cipher. We were given the first two characters of the
intermediate text --`Hg'.

To decrypt the message, the two stages had to be performed in reverse order:
first the XOR stage, followed by the unknown cipher stage. 

I began by creating a python script that first reads in the secret message, and
then retrieves the key for the known portion of the message. Given that we only
had two letters of the intermediate text, I guessed that the key length of the
message is two, and repreated the key that I had retrieved for the length of
the message to retrieve some text. The Python code that I used to decode the
message into it's intermediate form is as follows:

\begin{python}
with open("secret.hex") as f:
    ciphertext = f.read()

    k1 = chr(ord(ciphertext[0:1]) ^ ord("H"))
    k2 = chr(ord(ciphertext[1:2]) ^ ord("g"))
    keystream = (k1 + k2) * (len(ciphertext) / 2)

    plaintext = "".join(
            [chr(ord(k) ^ ord(c)) 
             for k, c in zip(keystream, ciphertext)]
    )

    print plaintext
\end{python}

This resulted in the followin intermediate text; 

\begin{lstlisting}
Hgt ltsbuba pm dumt uy hgt lpyh wqpmpxbr sbr tdxyuct lvyhtqv
pm hgtl sdd, xbobpib hp tctb hgt aqtshtyh lubry.  Yxqtdv
sbvpbt igp htddy vpx hgtv gsct hgt sbyitq uy npouba, lsr pq
yulwdv luyhsotb.  Hgtqt sqt lsbv hgubay hgsh lsot dumt ipqhg
gpdruba pb hp sbr yscpxquba. Exh dumt uy xbwqtrukhsedt sbr it
sqt pmhtb lvyhtquty tctb hp pxqytdcty.  It hgubo yxkktyy,
gswwubtyy, gtdwuba phgtqy, pq yxqwsyyuba pxqytdcty iudd lsot
dumt ipqhg ducuba, exh it ksb sdisvy et iqpba pq mqxyhqshtr ev
tctbhy. Hguy uy s qsbrpl wgqsyt.  Wgudpypwgtqy gsct s dph hp
ysv sepxh hgt csdxt pm sdd hgtyt hgubay, sbr s duhhdt dtyy hp
ysv sepxh pbt pm hgt lpyh csdxsedt hgubay pm sdd: dpct. Yp it
ksb et kdtsq tbpxag sepxh igsh uh ltsby mpq dumt hp gsct
ltsbuba sbr csdxt, exh igtb it wxh rpib pxq wgudpypwgv eppoy
sbr skhxsddv ath pb iuhg ducuba, ltsbuba sbr csdxt ksb et
tdxyuct. Ducuba itdd uy lpqt sqh hgsb ykutbkt pq wgudpypwgv.
Hgtqtmpqt,  hgt pbdv ytbyt it ksb lsot pm hgt urts hgsh dumt
gsy ltsbuba uy hgsh hgtqt sqt yplt qtsypby hp duct qshgtq hgsb
hp rut, sbr hgpyt qtsypby sqt hp et mpxbr ub hgt ducuba pm
dumt uhytdm.
\end{lstlisting}

By simple inspection, this looked like it was likely to be the output of a
basic cipher -- the message contains only alphabetic characters, and there are
spaces evenly distributed throughout the text. Although this could have just
been by chance, it definitely looked like a good lead to follow.

As I believed this text looked like human language, I decided the next step was to calculate the Index of Coincidence (IC) for this text. The IC is a weighted sum of the frequency of the letters, and can be used to identify if a ciphertext has the same frequency distribution as a given language. If it has the same distribution, then the cipher is a simple substitution cipher. The IC is calculated as:

$$
\textup{IC} =  \frac{\sum_{i=1}^{c} n_i(n_i -1)}{N(N - 1)/c}
$$

Where $N$ is the length of the text. $n_1$ to $n_c$ are the counts
for each of the possible characters. And $c$ is the number of
letters in the alphabet. 
 
\section{Headline Puzzles}
This question has been divided into five parts, labeled 1 to 5.

I found this question more challenging than first expected because:
\begin{itemize}
\item Ciphertext not long enough to perform frequency analysis
\item Keyspace is too large to brute force
\item We cannot be certain of the encryption mechanism
\item Some of the texts do not contain any double characters (e.g. "oo")
\end{itemize}

% references section

% can use a bibliography generated by BibTeX as a .bbl file
% BibTeX documentation can be easily obtained at:
% http://www.ctan.org/tex-archive/biblio/bibtex/contrib/doc/
% The IEEEtran BibTeX style support page is at:
% http://www.michaelshell.org/tex/ieeetran/bibtex/
%\bibliographystyle{IEEEtran}
% argument is your BibTeX string definitions and bibliography database(s)
%\bibliography{IEEEabrv,../bib/paper}
%
% <OR> manually copy in the resultant .bbl file
% set second argument of \begin to the number of references
% (used to reserve space for the reference number labels box)
%\begin{thebibliography}{1}
%\end{thebibliography}


%\vfill

% that's all folks
\end{document}
