\documentclass[11pt,a4paper,twoside]{article}
%\documentclass[journal]{IEEEtran}

\usepackage[stable]{footmisc}
\hyphenation{op-tical net-works semi-conduc-tor}

\usepackage[top=2.2cm, bottom=2.2cm, left=2.4cm, right=2.4cm]{geometry} 

\usepackage{listings}
\lstset{basicstyle=\ttfamily, escapechar=\€}

\usepackage{color}
\usepackage{setspace}

\usepackage{amsfonts}
\usepackage{amsmath}

\usepackage{booktabs}

\usepackage{float}
\restylefloat{table}

\usepackage{titlesec}

\titlespacing*{\section}{0pt}{2pt}{4pt}
\titlespacing*{\subsection}{0pt}{2pt}{4pt}


\definecolor{Code}{rgb}{0,0,0}
\definecolor{Decorators}{rgb}{0.5,0.5,0.5}
\definecolor{Numbers}{rgb}{0.5,0,0}
\definecolor{MatchingBrackets}{rgb}{0.25,0.5,0.5}
\definecolor{Keywords}{rgb}{0,0,1}
\definecolor{self}{rgb}{0,0,0}
\definecolor{Strings}{rgb}{0,0.63,0}
\definecolor{Comments}{rgb}{0,0.63,1}
\definecolor{Backquotes}{rgb}{0,0,0}
\definecolor{Classname}{rgb}{0,0,0}
\definecolor{FunctionName}{rgb}{0,0,0}
\definecolor{Operators}{rgb}{0,0,0}
\definecolor{Background}{rgb}{0.98,0.98,0.98}

\lstnewenvironment{python}[1][]{
\lstset{
numbers=left,
numberstyle=\footnotesize,
numbersep=1em,
xleftmargin=1em,
framextopmargin=2em,
framexbottommargin=2em,
showspaces=false,
showtabs=false,
showstringspaces=false,
frame=l,
tabsize=4,
% Basic
basicstyle=\ttfamily\small\setstretch{1},
backgroundcolor=\color{Background},
language=Python,
% Comments
commentstyle=\color{Comments}\slshape,
% Strings
stringstyle=\color{Strings},
morecomment=[s][\color{Strings}]{"""}{"""},
morecomment=[s][\color{Strings}]{'''}{'''},
% keywords
morekeywords={import,from,class,def,for,while,if,is,in,elif,else,not,and,or,print,break,continue,return,True,False,None,access,as,,del,except,exec,finally,global,import,lambda,pass,print,raise,try,assert},
keywordstyle={\color{Keywords}\bfseries},
% additional keywords
morekeywords={[2]@invariant},
keywordstyle={[2]\color{Decorators}\slshape},
emph={self},
emphstyle={\color{self}\slshape},
%
}}{}

\usepackage{paralist}
\usepackage{enumitem}

\begin{document}

\title{Cryptography: Secrecy Through Numbers}
\author{Edward~JF~Seabrook}

\maketitle

\section{Elliptic Curves}
\subsection{Part a}
The Elliptic curve I was assigned was $y^2 = x^3 + 7x + 1$, over the field
$\mathbb{F}_{17}$. I began by calculating all the values of $y^2$:

\begin{table}[H]
\centering
\begin{tabular}{l|lllllllllllllllll}
$y$   & 0 & 1 & 2 & 3 & 4  & 5 & 6 & 7  & 8  & 9  & 10 & 11 & 12 & 13 & 14 & 15 & 16 \\
$y^2$ & 0 & 1 & 4 & 9 & 16 & 8 & 2 & 15 & 13 & 13 & 15 & 2  & 8  & 16 & 9  & 4  & 1 
\end{tabular}
\end{table}

Then, using two lines of python:

\begin{python}
for x in range(0, 16):
    print (x**3 + 7*x + 1) % 17 
\end{python}

I produced a table of $y^2$ values for each value of $x$, which I mapped back to $y$ values.\

\begin{table}[h]
\centering
\begin{tabular}{lllllllll}
$x$ & $y^2$ & $y$     & & & &  $x$ & $y^2$ & $y$     \\  \midrule
0   & 1     & 1 or 16 & & & & 9   & 11    & --      \\
1   & 9     & 3 or 14 & & & & 10  & 0     & 0       \\
2   & 6     & --      & & & & 11  & 15    & 10 or 7 \\
3   & 15    & 10 or 7 & & & & 12  & 11    & --      \\
4   & 8     & 5 or 12 & & & & 13  & 11    & --      \\
5   & 8     & 5 or 12 & & & & 14  & 4     & 15 or 2 \\
6   & 4     & 15 or 2 & & & & 15  & 13    & 8 or 9  \\
7   & 2     & 6 or 11 & & & & 16  & 10    & -- \\
8   & 8     & 5 or 12 
\end{tabular}
\end{table}

The points on my curve are: (0,1), (0,16), (1, 3), (1, 14), (3, 10), (3, 7),
(4, 5), (4, 12), (5, 5), (5,12), (6, 15), (6, 2), (7, 6), (8, 5), (8, 12), (10,
0), (11, 10), (11,7), (14, 5), (14, 2), (15, 8), (15, 9) and the point at
infinity. The order of my group is twenty four. 

\subsection{Part b}
The group has order of twenty four. A cyclic group is the direct product of two
smaller cyclic groups. My group can have one of the following structues:
$\mathbb{C}_{24}$, $\mathbb{C}_{12}\times\mathbb{C}_{2}$,
$\mathbb{C}_{6}\times\mathbb{C}_{4}$, or $\mathbb{C}_{3}\times\mathbb{C}_{8}$

For a group to be equivilent to the group I produced, it must have exactly one
element that has order two, because there is only one element that when added
to itself produces the identity, a point with $y = 0$. I enumerated the groups
by hand to find that they have the following number of elements of orders one
and two: 

\begin{table}[H]
\centering
\begin{tabular}{l@{}cc}
Group                                 & Elements of Order One & Elements of Order Two \\  \midrule
$\mathbb{C}_{24}$                     & 1                     & 1                     \\
$\mathbb{C}_{12}\times\mathbb{C}_{2}$ & 1                     & 3                     \\
$\mathbb{C}_{6}\times\mathbb{C}_{4}$  & 1                     & 3                     \\
$\mathbb{C}_{3}\times\mathbb{C}_{8}$  & 1                     & 1                     
\end{tabular}
\end{table}

This means that my group must be either $\mathbb{C}_{24}$ or
$\mathbb{C}_{3}\times\mathbb{C}_{8}$. $\mathbb{C}_{24}$ is cyclic, because it
is the cyclic group of 24 elements. $\mathbb{C}_{3}\times\mathbb{C}_{8}$ is the
product of two groups whose sizes are coprime, this means that it is also
cyclic. Since both these groups are cyclic, my group must be cyclic. The answer
to the original question, is the group cyclic, is yes. 

\subsection{Part c}
Elliptic curves have the interesting property of that any two points on the
line will intersect the line again at another point. This can be used as a
group operator, and used to generate a group. The group operation is a
relatively cheap operation to perform, so can be applied multiple times without
consuming large amounts of CPU time. 

The invese of the group operation, finding the discrete logarithm of a point on
an elliptic curve, for a known base, is believed to be a hard problem that
cannot be feasibly solved. Due to this `trapdoor' proprty, elliptic curves can
be used in a number of public key algorithms, including: elliptic curve
Diffie–Hellman (ECDH), the Elliptic Curve Integrated Encryption Scheme (ECIES),
and the Elliptic Curve Digital Signature Algorithm (ECDSA). ECDH is explained
and demonstrated in Part d.

Elliptic curves provide stronger security for shorter key lengths that other
traditional public-key cryptography, such as that of the Integers modulo a
large prime. 

\subsection{Part d}
To encrypt the message: “A little knowledge is a dangerous thing”. I used
elliptic curve Diffie-Hellman to produce a session key, and then used the RC4
stream cipher for the encryption of the message. 

I began by writing code that performs the group operation on points on the
elliptic curve, the code for this can be found in Appendix \ref{ECC}.  

Next, I wrote a script that simulates the Diffie-Hellman key exchange, using
the elliptic curve functions provided by the code from the first part. The
code, found in Appendix \ref{ECDH}, uses the following alogrithm:

\begin{enumerate}[noitemsep]
\item Alice chooses parameters for the elliptic curve. i.e. A, B and P. These are all sent to Bob.
\item Alice picks a random point on the curve, $g$, and sends this to Bob.
\item Alice and Bob then independently pick random numbers: $K_{a}$ and $K_{b}$. 
\item $K_{a}$ and $K_{b}$ are kept secret, and are both less than the order of the group. 
\item Alice calculates $g^{K_{a}}$ and sends this to Bob.
\item Bob calculates $g^{K_{b}}$ and sends this to Alice. 
\item Alice then calculates $(g^{K_{b}})^{K_{a}}$ which equals $g^{K_{a}K_{b}}$.
\item Bob then calculates $(g^{K_{a}})^{K_{b}}$ which equals $g^{K_{a}K_{b}}$.
\item Both parties then take the $x$ component of the shared point $g^{K_{a}K_{b}}$ to be the shared key.
\end{enumerate}

This algoithm results in a secret shared between the both parties, which no
other party could calculate by eavesdropping on the communications. 

I performed ECDH using my curve above, and the values of $g = (11, 7)$, $K_{a}
= 5$, and  $K_{b} = 6$. This resulted in the shared point (12, 7). Hence the
shared key was ``12''. 

Next, I encrypted this message with an RC4 implementation that I found
online\footnote{http://code.activestate.com/recipes/576736-rc4-arc4-arcfour-algorithm/},
that I adapted to use Base64 encoding to allow the ciphertext to be displayed
in this document. This resulted  in the ciphertext:
``HkyIm1ppneZOBbm1XC2xtCnRBGHVFJvVQzOM4wr24WkLJC8xFZhS''. I also decrypted the
message with the same key, and was able to retrieve the original message again. 

It is important to note that this particular key exchange is not secure,
because the size of the group used is very small. If this were to be used in
the real world, a far larger group should be used. 

\vfill
\pagebreak
\section{Vigenere Cipher}
We were given the following message, and instructed to decrypt it:

\begin{quote}
\small
\raggedright
qhc jeqpaeb srxrrp hcoe pbccktjv hyoduxrc qrmgalp hyse yqtpxcrbd ree yqtcktgln
mc gmsepkmcktq xnb oeqbapzhcos mke mc tfb myfn alnabrlp iq qhyq ilqeeoarbd
afrarirp il jijftyoy mo cpftgzaj fndoaqqrsztsoe ympjfcyqimks aluja bc
jajfcgluqiy kxngmujxtca dsoild tfb mykudxcrrrgkg nooabsq thgzh mctck tyheq
mlyze yyrmxd flwcsep pilze reepb hyse zbel ko pbpmotca hyoduxrc qrmgalp il
mryztgze wbt jftrie gp kllwl xbmrt flw qrcf x tpljyk wmrlb iomh lghe ykd flw
fxrb qo gjpjbmckt mke gk ppxcrfcc lnc bxyjpjb iq aonxnr qrmgal qhgp cyk bc rsca
tm zokmrmjiqb tfb sczupftw lf y jeykildfsi rcxl ulrja tyogcq wfflc xvmfdgkg
bbtcztgln zv fskcrfolxl rbsrfne xs ublj xs roohxn bbtcztgln kbcfxngpmq puae
tpljyks axn zb uqbd rl eqqaziiqe a ffdbbn qfdc zhyknci il xn mqhcowgpe qfdc
zhyknci rcpiqqalq dcpiek tffs roohxn bleq kor zhykgc qhc ioefc txlsb od xnw
darb bsq ilptcxd aealdeq lnjv tfb pmtep mrmcijb od qwm darbs yk etxlsxtmo wfl
iq kor xwyoe mc tfb tpljyk cyknmq arqaah tfb tpljyk dcpiek uqfne zokjol pibb
cfxnlbl yqtyzkq qhc lwlbr mc tfb tpljyk hmtetbr axn spe ffs ikouiebde mc tfb
tpljyk pmtep jobbl rl eqqaziiqe a ffdbbn qfdc zhyknci tfxt pblgxbjv lcxkq lur
peaoer hewp tffs gp tfb ela od qhc jeqpaeb 
\end{quote}

My initial guess was that this cipher was a monoalphabetic substitution cipher,
since the ciphertext appears to have spaces in roughly the correct places for
it to be natural language. I started solving this cipher by calculating the
Index of Coincidence (IC) for the text. The IC is a weighted sum of the
frequency of the letters, which can be used to determine if a ciphertext has
the same frequency distribution as a given language. If the distribution is
correct, then it is likely the cipher is a simple monoalphabetic substitution
cipher. The IC is calculated as:

$$
\textup{IC} =  \frac{\sum_{i=1}^{c} n_i(n_i -1)}{N(N - 1)/c}
$$

Where $N$ is the length of the text. $n_1$ to $n_c$ are the counts for each of
the possible characters. And $c$ is the number of letters in the alphabet. The
code I wrote to calculate this can be found in Appendix \ref{IC_code}. I
obtained a value of 1.128, which is quite different from the IC for English of
1.73. From this I concluded that the cipher used is not a simple monoalphabetic
substitution. 

I then iterated through different strides to see if the cipher used was
dependent on the position in the text. A stride of 2 did not produce a better
result, yielding 1.126 and 1.129 starting from characters 0 and 1 respectively.  

Using a stride of 3 returned a better set of values: 1.844, 1.759 and 1.80; for
starting character positions of 0, 1 and 2 respectively. This indicated that it
was worth investigating. I began by treating the cipher as a Vigenere cipher,
with a key length of three. On inspection, the triad 'qhc' jumped out at me as
a repeating pattern. Due to the commonness of the word, I figured that the
word could be `the'. If this was correct, then the Vigenere key would be 'xay',
resulting in a shifts of 23, 0 and 2.

Using the following python code seen in Appendix \ref{vig_code}, I generated
the answer: 

\begin{quote}
\small
\raggedright
the message starts here recently hardware trojans have attracted the attention
of governments and researchers one of the main concerns is that integrated
circuits in military or critical infrastructure applications could be
maliciously manipulated during the manufacturing process which often takes
place abroad however since there have been no reported hardware trojans in
practice yet little is known about how such a trojan would look like and how
hard to implement one in practice one example is dopant trojan this can be used
to compromise the security of a meaningful real world target while avoiding
detection by functional testing as well as trojan detection mechanisms such
trojans can be used to establish a hidden side channel in an otherwise side
channel resistant design this trojan does not change the logic value of any
gate but instead changes only the power profile of two gates an evaluator who
is not aware of the trojan cannot attack the trojan design using common side
channel attacks the owner of the trojan however can use his knowledge of the
trojan power model to establish a hidden side channel that reliably leaks out
secret keys this is the end of the message
\end{quote}
 


\vfill
\pagebreak
\section{Two Part Encryption}
The file ``secret.hex'' contained a secret message. The message had been
encrypted in two stages, the first stage was unknown, and the second stage was
a simple XOR based cipher. We were given the first two characters of the
intermediate text --`Hg'. To decrypt the message, the two stages had to be
performed in reverse order.

I began by creating a python script that reads in the secret message, and
retrieves the key for the known portion of the message. Given that we only
had two letters of the intermediate text, I guessed that the key length of the
message is two, and repeated the key that I had retrieved for the length of
the message to retrieve some text. The code I wrote to decode the
message into its intermediate form is in Appendix \ref{XOR_code}.


I managed to retrieve the following intermediate text: 

\begin{quote}
\small
\raggedright
Hgt ltsbuba pm dumt uy hgt lpyh wqpmpxbr sbr tdxyuct lvyhtqv pm hgtl sdd,
xbobpib hp tctb hgt aqtshtyh lubry.  Yxqtdv sbvpbt igp htddy vpx hgtv gsct hgt
sbyitq uy npouba, lsr pq yulwdv luyhsotb.  Hgtqt sqt lsbv hgubay hgsh lsot dumt
ipqhg gpdruba pb hp sbr yscpxquba. Exh dumt uy xbwqtrukhsedt sbr it sqt pmhtb
lvyhtquty tctb hp pxqytdcty.  It hgubo yxkktyy, gswwubtyy, gtdwuba phgtqy, pq
yxqwsyyuba pxqytdcty iudd lsot dumt ipqhg ducuba, exh it ksb sdisvy et iqpba pq
mqxyhqshtr ev tctbhy. Hguy uy s qsbrpl wgqsyt.  Wgudpypwgtqy gsct s dph hp ysv
sepxh hgt csdxt pm sdd hgtyt hgubay, sbr s duhhdt dtyy hp ysv sepxh pbt pm hgt
lpyh csdxsedt hgubay pm sdd: dpct. Yp it ksb et kdtsq tbpxag sepxh igsh uh
ltsby mpq dumt hp gsct ltsbuba sbr csdxt, exh igtb it wxh rpib pxq wgudpypwgv
eppoy sbr skhxsddv ath pb iuhg ducuba, ltsbuba sbr csdxt ksb et tdxyuct. Ducuba
itdd uy lpqt sqh hgsb ykutbkt pq wgudpypwgv.  Hgtqtmpqt,  hgt pbdv ytbyt it ksb
lsot pm hgt urts hgsh dumt gsy ltsbuba uy hgsh hgtqt sqt yplt qtsypby hp duct
qshgtq hgsb hp rut, sbr hgpyt qtsypby sqt hp et mpxbr ub hgt ducuba pm dumt
uhytdm.
\end{quote}

By inspection, this looked like the output of a basic substitution cipher --
the message contains only alphabetic characters, and there are spaces evenly
distributed throughout the text.

I calculate a value of 1.7086 for the IC, using the code in Appendix
\ref{IC_code}.  This is very close to that of the English language (1.73). This
finding confirmed my belief that the cipher used was some form of
monoalphabetic substitution.  

By copying the intermediate text into {\tt vim}, and using the search and
replace functionality (s/x/Y/g), I began substituting letters with likely
substitutions. I based my substitutions on the frequencies of the
letters\footnote{These frequencies were calculated as a by product of the
calculation of the IC. }; i.e. beginning by replacing `t' ,the most common
letter in the text, with `E', the most common letter in English. Once I had
replaced a few letters, it became obvious what words in the text could
potentially be, so I based my substitutions on this. Eventually, I managed to
obtain the following message: 

\begin{quote}
\small
\raggedright
The meaning of life is the most profound and elusive mystery of them all,
unknown to even the greatest minds. Surely anyone who tells you they have the
answer is joking, mad or simply mistaken.  There are many things that make life
worth holding on to and savouring. But life is unpredictable and we are often
mysteries even to ourselves. We think success, happiness, helping others, or
surpassing ourselves will make life worth living, but we can always be wrong or
frustrated by events.  This is a random phrase.  Philosophers have a lot to say
about the value of all these things, and a little less to say about one of the
most valuable things of all: love. So we can be clear enough about what it
means for life to have meaning and value, but when we put down our philosophy
books and actually get on with living, meaning and value can be elusive. Living
well is more art than science or philosophy. Therefore,  the only sense we can
make of the idea that life has meaning is that there are some reasons to live
rather than to die, and those reasons are to be found in the living of life
itself.
\end{quote}

\vfill
\pagebreak
\section{Headline Puzzles}
The headline puzzles were a cryptographic puzzle based around five newspaper
headlines from around the time where the puzzle was published. We were asked to
decrypt the following set of headlines: 

\begin{enumerate}[noitemsep]
\item YNTS QHABT YBK KJVT NR ORLSJN HCTCYA HQYKJV CYOCMBYNT 
\item GXRYK SXRKVWNRNIO YJVONHB NH VH KXASH OAXBBJNHB WNHB 
\item KSXXMT, FVTS SVJYMBF CFI EI BNSYYC JTMKEID 
\item AXITUL PGGTXLW VGA OCXFT AUMCAL VAGH RXDKQPUR PXDM
\item HQUSESTYY TBDSPKTTY YTT ERYHURBRWCVRPW RW JCBRSKJURTWESK DPSRHRTY
\end{enumerate}

I began solving this problem by writing a python script to help me decode some
of the headlines. The program, found in Appendix \ref{headline_code}, scans
through a sentence, and generates the patterns of letters of the words. These
patterns are then compared to the patterns of words in a 100,000 word English
dictionary\footnote{http://www.karamasoft.com/ultimatespell/dictionary.aspx}.
It returns as a result the words, and their potential matches sorted in order
of least matches. I also extended the script to treat letters in lowercase as
known letters, and restrict the guesses to words that meat this constraint.  

The script returned only one match for `ERYHURBRWCVRP': `discrimination'.  From
here, many of the letters in this puzzle could already be determined. I
substituted this word back into my script, and the words 'policies',
`employees', `see' and `in' were the only choices. Substituting these new words
back in, resulted in the phrase: `cQrldless employees see discrimination in
JamilyJriendly policies'

Although my dictionary didn't contain the phrase `familyfriendly', I was able
to work this out using intuition.`cQrldless' didn't seem to have any matching
words. It appears that there has been an error here, and given the context the
correct plain text word should be `childless'. 

Using this same method I was able to deduce the other messages as:
\begin{itemize}[noitemsep]
\item Puzzle 1: ``ntsb urges new ways to combat rising runway incidents''
\item Puzzle 2: ``dutch authorities closing in on human smuggling ring''
\item Puzzle 5: ``childless employees see discrimination in familyfriendly policies''
\end{itemize}

Puzzles 3 \& 4 lacked long words with few enough possible answers to make any
educated guesses. As I already had more than two decoded puzzles to work with,
I left these unanswered.  

The way that headline puzzles are constructed uses a permuted alphabet that is
offset against itself. This means that you can build chains of letters that can
be used to discover an equivalent alphabet that can be used to decode all the
messages, when the correct offsets are applied to it. I constructed this table
of letters to help me decode the message. 

\begin{lstlisting}
   a b c d e f g h i j k l m n o p q r s t u v w x y z
1. J S O M B   A   C       L Y R     H T N Q   K   V   
2. S   Y G I   B K N     J A H V     W O R X           
3.                                                     
4.                                                     
5. C   H E T J   Q R     S B W P D   U Y V         K   
   a b c d e f g h i j k l m n o p q r s t u v w x y z
\end{lstlisting}

I built up the following chains by setting the substitutions against one
another. I had to use letters uncovered in later matrices to complete any of
the chains. 

\pagebreak
\subsection{Headline One vs Five}
This was the longest chain that I built up, I did hit several roadblocks that
were only solvable once I had learned new substitutions from other chains. 

\begin{lstlisting}
                  Z
                  X
                  F J O P D E T V 
                  G I R U M B N W  
                  A C H Q L S Y K 
                F J O P D E T V 
                G I R U M B N W  
                A C H Q L S Y K 
              F J O P D E T V 
              G I R U M B N W
              A C H Q L S Y K
            F J O P D E T V 
            G I R U M B N W 
            A C H Q L S Y K 
          F J O P D E T V  
      ... G I R U M B N W X A C H Q L S Y K F J O P D E T V Z G ...
          A C H Q L S Y K F J
        F J O P D E T V Z G    
        G I R U M B N W X 
        A C H Q L S Y K
      F J O P D E T V 
      G I R U M B N W  
    X A C H Q L S Y K
    F J O P D E T V 
    G I R U M B N W  
    A C H Q L S Y K
                  Z
\end{lstlisting}

\subsection{Headline One vs Two}
This matrix was very useful for finding more substitutions for use in the
previous chain. 
\begin{lstlisting}
      U Q D M L E B S T N Y V W K Z X F G A J I C O R H P U Q D M
      X F G A J I C O R H P U Q D M L E B S T N Y V W K Z X F G A 
      L E B S T N Y V W K Z X F G A J I C O R H P U Q D M L E B S       
\end{lstlisting}

\subsection{Headline Two vs Five}
This matrix helped me to find the very last few letters; such as `Z'  
\begin{lstlisting}
              R U M B N W X A C H Q L S Y K F J O P D E T V Z G 
    R U M B N W X A C H Q L S Y K F J O P D E T V Z G
\end{lstlisting}


From these chains I was able to build up this substitution table.
\begin{lstlisting}
   a b c d e f g h i j k l m n o p q r s t u v w x y z
1. J S O M B G A P C I Z E L Y R U D H T N Q W K F V X 
2. S C Y G I E B K N T D J A H V Z F W O R X U Q L P M 
3.                                                     
4.                                                     
5. C N H E T J I Q R O F S B W P D L U Y V M Z X A K G 
   a b c d e f g h i j k l m n o p q r s t u v w x y z
\end{lstlisting}

\pagebreak

Using the chain uncovered in Headline One vs Two, I found the offsets that
produced the correct plaintext for each headline. To help me do this, I wrote a
script, found in Appendix \ref{Headline_checker}, to enumerate all of the
possible rotations of the key. I scanned this list my hand looking for phrases
that were readable English. This resulted in the following offsets of the
alphabet. 

\begin{lstlisting}
k LEBSTNYVWKZXFGAJICORHPUQDM
  --------------------------
1 EBSTNYVWKZXFGAJICORHPUQDML
2 JICORHPUQDMLEBSTNYVWKZXFGA
3 XFGAJICORHPUQDMLEBSTNYVWKZ
4 PUQDMLEBSTNYVWKZXFGAJICORH
5 STNYVWKZXFGAJICORHPUQDMLEB
\end{lstlisting}

My final answers for this question were: 

\begin{enumerate}[noitemsep]
\item NTSB URGES NEW WAYS TO COMBAT RISING RUNWAY INCIDENTS
\item DUTCH AUTHORITIES CLOSING IN ON HUMAN SMUGGLING RING
\item DOLLAR EURO OUTPACE YEN IN CHOPPY TRADING
\item RIPKEN LOOKING FOR QUICK RETURN FROM DISABLED LIST
\item CHILDLESS EMPLOYEES SEE DISCRIMINATION IN FAMILYFRIENDLY POLICIES
\end{enumerate}

Once the correct answers have been obtained, it was possible to search the Web,
and retrieve the correct answers from an NSA leaflet. The NSA solution goes on
to uncover the Setting: TOADY, Key: SYCOPHANT and Hat: BOOTLICKER

% references section

% can use a bibliography generated by BibTeX as a .bbl file
% BibTeX documentation can be easily obtained at:
% http://www.ctan.org/tex-archive/biblio/bibtex/contrib/doc/
% The IEEEtran BibTeX style support page is at:
% http://www.michaelshell.org/tex/ieeetran/bibtex/
%\bibliographystyle{IEEEtran}
% argument is your BibTeX string definitions and bibliography database(s)
%\bibliography{IEEEabrv,../bib/paper}
%
% <OR> manually copy in the resultant .bbl file
% set second argument of \begin to the number of references
% (used to reserve space for the reference number labels box)
%\begin{thebibliography}{1}
%\end{thebibliography}

\vfill
\pagebreak
\section*{Appendix} 
\appendix 

All code found in this appendix was written by me, and should be executed using
Python 2.7. 

\section{Elliptic Curve Code}
\label{ECC}
This code calculates the elliptic curve operator, for point addition and
multiplication. There are additional functions to simplify common tasks such as
raising to a power. 
\begin{python}
A = 7
B = 1
P = 17

def mult_inv(x):
    #Exploit special case because P is prime
    return x**(P-2) % P

class Point:
    def __init__(self, x, y):
        self.x = x
        self.y = y
    def __str__(self):
        return "({}, {})".format(self.x, self.y)
    
def double(point):
    lam = ((3 * point.x**2 + A) * mult_inv(2 * point.y)) % P 
    new_x = (lam**2 - 2*point.x) % P
    new_y = (lam * (point.x - new_x) - point.y) % P
    p = Point(new_x, new_y)
    return p

def add(p, q):
    lam = ((q.y - p.y) * mult_inv(q.x- p.x)) % P
    new_x = (lam**2 - p.x - q.x) % P
    new_y = (lam * (p.x - new_x) - p.y) % P
    p = Point(new_x, new_y)
    return p

def multiple_add(p, q, n):
    for i in range(1, n):
        q = add(p, q)
    return q

def power(p, n):
    q = double(p)
    for i in range(2, n):
        q = add(p, q)
    return q
\end{python}
\vfill
\pagebreak

\section{EC Diffie-Hellman}
\label{ECDH}
This code uses the Elliptic curve functions defined above to simulate the
Diffie-Hellman key exchange. 
\begin{python}
g = Point(11,7)

#A 
A_key = 5
gA = power(g, A_key)

#B 
B_key = 6
gB = power(g, B_key)

#A sends gA and B sends gB to A

#A 
gAB = multiple_add(g, gB, A_key)
print gAB

#B 
gBA = multiple_add(g, gA, B_key)
print gBA
\end{python}

\section{Vigenere Code}
\label{vig_code}
This code is used to solve a Vigenere cipher where provided with the correct
key. 
\begin{python}
import string

def shift(c, key):
    n = ord(c) - ord('a')
    k = ord(key) - ord('a')
    return chr((n - k) % 26 + ord('a'))

key = "xay"
answer = ""

with open('ciphertext.txt'):
    n = 0
    for c in s:
        if c in string.ascii_lowercase:
            answer += shift(c, key[n%len(key)])
            n += 1
        else:
            answer += c
        
print answer
\end{python}

\vfill
\pagebreak

\section{IC Code}
\label{IC_code}
This code is used to generate the Index of Coincidence for a text. It can also
be used to count letter frequencies. 
\begin{python}
from __future__ import division
import string

with open("intermediate.txt") as f: 
    s = f.read()

    freq = {}

    for c in string.ascii_uppercase:
        freq[c.upper()] = 0

    for c in s:
        if c.upper() in string.ascii_uppercase:
            freq[c.upper()] += 1
                           
    N = sum([n for n in freq.values()])
    c = len(string.ascii_uppercase)

    numerator = sum([(n * (n-1)) for n in freq.values()]) 
    denominator = ((N * (N-1))/c)

    IC = numerator / denominator 

    print IC
\end{python}

\section{XOR Code}
\label{XOR_code}
This code solves the XOR part of the two part cryptography section. 
\begin{python}
with open("secret.hex") as f:
    ciphertext = f.read()

    k1 = chr(ord(ciphertext[0:1]) ^ ord("H"))
    k2 = chr(ord(ciphertext[1:2]) ^ ord("g"))
    keystream = (k1 + k2) * (len(ciphertext) / 2)

    plaintext = "".join(
            [chr(ord(k) ^ ord(c)) 
             for k, c in zip(keystream, ciphertext)]
    )

    print plaintext
\end{python}

\vfill
\pagebreak
\section{Headline Code}
\label{headline_code}
This code is used to find possible words based on the patterns of words in
cipher text. 
\begin{python}
import string


raw_sentence =   """YNTS QHABT YBK KJVT NR ORLSJN 
                 HCTCYA HQYKJV CYOCMBYNT"""
known_sentence = """YNTS QHABT YBK KJVT NR ORLSJN 
                 HCTCYA HQYKJV incident"""


def pattern(s):
    seen = {}
    out = ""
    x = 'a'
    for c in s:
        if c in seen.keys():
            out += seen[c]
        else:
            out += x
            seen[c] = x
            x = chr(ord(x)+1)
    return out

def words_match(a, b):
    for c, d in zip(a, b):
        if c in string.ascii_lowercase:
            if c != d:
                return False
    return True

def find_word(word, pattern_dic):
    word_list = []
    for i, (w, p) in enumerate(pattern_dict):
        if p == pattern(word) and words_match(word, w):
            word_list.append(w)
    return word_list

def expand_known_characters(raw, known):
    out = raw
    for r,k in zip(raw, known): 
      if k in string.ascii_lowercase:
          out = string.replace(out, r, k)
    return out

with open("en-GB.dic") as word_list:
    dictionary = [w.lower() for w 
                  in word_list.read().split("\r\n") 
                  if "'" not in w]
    pattern_dict = [(w, pattern(w)) for w in dictionary]

sentence = expand_known_characters(raw_sentence, 
                                   known_sentence)
print "\"{}\"\n".format(sentence)

guesses = {}

for w in sentence.split(" "):
    guesses[w] = find_word(w, pattern_dict)

guesses = sorted(guesses.iteritems(), key=lambda x:len(x[1]))

for word in guesses:
    print "{} -- {}\n".format(word[0], word[1])
    total += len(word[1])
\end{python}

\section{Headline Key Checker}
\label{Headline_checker}
This code is used to enumerate the possible headlines, using all rotations of the key alphabet. 
\begin{python}
def rotate(s):
    return s[1:]+s[:1]

def sub(ciphertext, key, rotated_key):
    m = {}
    for c, p in zip(key, rotated_key):
        m[p] = c
    
    plaintext = ""
    for c in ciphertext:
        if c == ' ':
            plaintext += ' '
        else:
            plaintext += m[c]
    
    return "{} -- {}".format(plaintext, rotated_key)
    

def check_key(ciphertext, key):
    
    k = key
    
    for i in range(0, len(key)):
        k = rotate(k)
        
        print sub(ciphertext, key, k)


check_key("YNTS QHABT YBK KJVT NR ORLSJN HCTCYA HQYKJV CYOCMBYNT", 
          "LEBSTNYVWKZXFGAJICORHPUQDM")
\end{python}
%\vfill

% that's all folks
\end{document}
